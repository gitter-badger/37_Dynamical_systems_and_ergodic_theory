\documentclass[12pt]{article}
\usepackage{pmmeta}
\pmcanonicalname{PughsClosingLemma}
\pmcreated{2013-03-22 14:07:13}
\pmmodified{2013-03-22 14:07:13}
\pmowner{Koro}{127}
\pmmodifier{Koro}{127}
\pmtitle{Pugh's closing lemma}
\pmrecord{8}{35526}
\pmprivacy{1}
\pmauthor{Koro}{127}
\pmtype{Theorem}
\pmcomment{trigger rebuild}
\pmclassification{msc}{37C20}
\pmclassification{msc}{37C25}
\pmsynonym{closing lemma}{PughsClosingLemma}

% this is the default PlanetMath preamble.  as your knowledge
% of TeX increases, you will probably want to edit this, but
% it should be fine as is for beginners.

% almost certainly you want these
\usepackage{amssymb}
\usepackage{amsmath}
\usepackage{amsfonts}
\usepackage{mathrsfs}

% used for TeXing text within eps files
%\usepackage{psfrag}
% need this for including graphics (\includegraphics)
%\usepackage{graphicx}
% for neatly defining theorems and propositions
%\usepackage{amsthm}
% making logically defined graphics
%%%\usepackage{xypic}

% there are many more packages, add them here as you need them

% define commands here
\newcommand{\C}{\mathbb{C}}
\newcommand{\R}{\mathbb{R}}
\newcommand{\N}{\mathbb{N}}
\newcommand{\Z}{\mathbb{Z}}
\newcommand{\Per}{\operatorname{Per}}
\begin{document}
Let $f:M\to M$ be a $\mathcal{C}^1$ diffeomorphism of a compact smooth manifold $M$. Given a nonwandering point $x$ of $f$, there exists a diffeomorphism $g$ arbitrarily close to $f$ in the $\mathcal{C}^1$ topology of $\operatorname{Diff}^1(M)$ such that $x$ is a periodic point of $g$.

The analogous theorem holds when $x$ is a nonwandering point of a $\mathcal{C}^1$ flow on $M$.

\begin{thebibliography}{9}
\bibitem{Pugh} Pugh, C., \emph{An improved closing lemma and a general density theorem}, Amer. J. Math. \textbf{89} (1967).
\end{thebibliography}
%%%%%
%%%%%
\end{document}
