\documentclass[12pt]{article}
\usepackage{pmmeta}
\pmcanonicalname{Preinjectivity}
\pmcreated{2013-03-22 19:22:04}
\pmmodified{2013-03-22 19:22:04}
\pmowner{Ziosilvio}{18733}
\pmmodifier{Ziosilvio}{18733}
\pmtitle{pre-injectivity}
\pmrecord{4}{42321}
\pmprivacy{1}
\pmauthor{Ziosilvio}{18733}
\pmtype{Definition}
\pmcomment{trigger rebuild}
\pmclassification{msc}{37B15}
\pmclassification{msc}{68Q80}
\pmdefines{mutually erasable patterns (cellular automaton)}

% this is the default PlanetMath preamble.  as your knowledge
% of TeX increases, you will probably want to edit this, but
% it should be fine as is for beginners.

% almost certainly you want these
\usepackage{amssymb}
\usepackage{amsmath}
\usepackage{amsfonts}

% used for TeXing text within eps files
%\usepackage{psfrag}
% need this for including graphics (\includegraphics)
%\usepackage{graphicx}
% for neatly defining theorems and propositions
%\usepackage{amsthm}
% making logically defined graphics
%%%\usepackage{xypic}

% there are many more packages, add them here as you need them

% define commands here

\begin{document}
\newcommand{\Acal}{\ensuremath{\mathcal{A}}}
\newcommand{\goe}{\mathbb{GoE}}
\newcommand{\ie}{\textit{i.e.}}
\newcommand{\Neigh}{\ensuremath{\mathcal{N}}}
\newcommand{\restrict}[2]{\ensuremath{\left.{#1}\right|_{#2}}}

\newtheorem{lemma}{Lemma}

Let $X = \prod_{i \in I} X_i$ be a Cartesian product.
Call two elements $x,y \in X$ \emph{almost equal}
if the set $\Delta(x,y) = \{i \in I \mid x_i \neq y_i\}$ is finite.
A function $f : X \to X$ is said to be \emph{pre-injective}
if it sends distinct almost equal elements into distinct elements,
\ie, if $0 < |\Delta(x,y)| < \infty$ implies $f(x) \not= f(y)$.

If $X$ is finite, pre-injectivity is the same as injectivity;
in general, the latter implies the former, but not the other way around.
Moreover, it is not true in general
that a composition of pre-injective functions
is itself pre-injective.

A cellular automaton is said to be pre-injective if its global function is.
As cellular automata send almost equal configurations
into almost equal configurations,
the composition of two pre-injective cellular automata is pre-injective.

Pre-injectivity of cellular automata can be characterized
via mutually erasable patterns.
Given a finite subset $E$ of $G$,
two patterns
\begin{math}
p_1,p_2 : E \to Q
\end{math}
are \emph{mutually erasable} (briefly, m.e.)
for a cellular automaton
\begin{math}
\Acal = \langle Q, \Neigh, f \rangle
\end{math}
on $G$
if for any two configurations
\begin{math}
c_1,c_2 : G \to Q
\end{math}
such that
\begin{math}
\restrict{c_i}{E} = p_i
\end{math}
and
\begin{math}
\restrict{c_1}{G \setminus E} = \restrict{c_2}{G \setminus E}
\end{math}
one has $F_\Acal(c_1) = F_\Acal(c_2)$.

\begin{lemma} \label{lem:me}
For a cellular automaton
\begin{math}
\Acal = \langle Q, \Neigh, f \rangle,
\end{math}
the following are equivalent.
\begin{enumerate}
\item \label{it:me}
$\Acal$ has no mutually erasable patterns.
\item \label{it:ea}
$\Acal$ is pre-injective.
\end{enumerate}
\end{lemma}
\textit{Proof.}
It is immediate that the negation of point~\ref{it:me}
implies the negation of point~\ref{it:ea}.
So let
\begin{math}
c_1,c_2 : G \to Q
\end{math}
be two distinct almost equal configurations such that
\begin{math}
F_\Acal(c_1) = F_\Acal(c_2):
\end{math}
it is not restrictive to suppose that
$\Neigh$ is symmetric
(\ie, if $x \in \Neigh$ then $x^{-1} \in \Neigh$)
and $e$, the identity element of $G$, belongs to $\Neigh$.
Let $\Delta$ be a finite subset of $G$ such that
\begin{math}
\restrict{c_1}{G \setminus \Delta} = \restrict{c_2}{G \setminus \Delta},
\end{math}
and let
\begin{equation}
E = \Delta \Neigh^2
  = \{ g \in G
       \mid \exists z \in \Delta, u,v \in \Neigh
       \mid g = zuv
    \} \;:
\end{equation}
we shall prove that
\begin{math}
p_1 = \restrict{c_1}{E}
\end{math}
and
\begin{math}
p_2 = \restrict{c_2}{E}
\end{math}
are mutually erasable.
(They surely are distinct, since $\Delta \subseteq E$.)

So let $\gamma_1,\gamma_2 : G \to Q$ satisfy
\begin{math}
\restrict{\gamma_i}{E} = p_i
\end{math}
and
\begin{math}
\restrict{\gamma_2}{G \setminus E} = p_i = \restrict{\gamma_2}{G \setminus E}.
\end{math}
Let $z \in G$.
If $z \in \Delta\Neigh$,
then $F_\Acal(\gamma_1)(z) = F_\Acal(\gamma_2)(z)$,
because by construction
\begin{math}
\restrict{\gamma_i}{z\Neigh} = \restrict{c_i}{z\Neigh};
\end{math}
if
\begin{math}
z \in G \setminus \Delta\Neigh,
\end{math}
then $F_\Acal(\gamma_1)(z) = F_\Acal(\gamma_2)(z)$ as well,
because by construction
\begin{math}
\restrict{\gamma_1}{z\Neigh} = \restrict{\gamma_2}{z\Neigh}.
\end{math}
Since $\gamma_1$ and $\gamma_2$ are arbitrary,
$p_1$ and $p_2$ are mutually erasable.
\hfill $\Box$

\begin{thebibliography}{99}

Ceccherini-Silberstein, T. and Coornaert, M. (2010)
\textit{Cellular Automata and Groups.}
Springer Verlag.

\end{thebibliography}

%%%%%
%%%%%
\end{document}
