\documentclass[12pt]{article}
\usepackage{pmmeta}
\pmcanonicalname{ChaoticDynamicalSystem}
\pmcreated{2013-03-22 13:05:26}
\pmmodified{2013-03-22 13:05:26}
\pmowner{bshanks}{153}
\pmmodifier{bshanks}{153}
\pmtitle{chaotic dynamical system}
\pmrecord{15}{33507}
\pmprivacy{1}
\pmauthor{bshanks}{153}
\pmtype{Definition}
\pmcomment{trigger rebuild}
\pmclassification{msc}{37G99}
\pmsynonym{chaotic system}{ChaoticDynamicalSystem}
\pmsynonym{deterministic chaotic system}{ChaoticDynamicalSystem}
\pmsynonym{chaotic behavior}{ChaoticDynamicalSystem}
%\pmkeywords{dynamical system}
%\pmkeywords{aperiodic dynamic behavior}
%\pmkeywords{chaos}
%\pmkeywords{deterministic behaviors}
\pmrelated{DynamicalSystem}
\pmrelated{SystemDefinitions}

\endmetadata

% this is the default PlanetMath preamble.  as your knowledge
% of TeX increases, you will probably want to edit this, but
% it should be fine as is for beginners.

% almost certainly you want these
\usepackage{amssymb}
\usepackage{amsmath}
\usepackage{amsfonts}

% used for TeXing text within eps files
%\usepackage{psfrag}
% need this for including graphics (\includegraphics)
%\usepackage{graphicx}
% for neatly defining theorems and propositions
%\usepackage{amsthm}
% making logically defined graphics
%%%\usepackage{xypic}

% there are many more packages, add them here as you need them

% define commands here
\begin{document}
As Strogatz says in reference [1], ``No definition of the term chaos is universally accepted yet, but almost everyone would agree on the three ingredients used in the following working definition''.

Chaos is the aperiodic long-term \PMlinkescapetext{behavior} in a deterministic system that exhibits sensitive dependence on initial conditions.

Aperiodic long-term \PMlinkescapetext{behavior} means that there are trajectories which do not settle down to fixed points, periodic \PMlinkname{orbits}{Orbit}, or quasiperiodic \PMlinkescapetext{orbits} as $t \to \infty$. For the purposes of this definition, a trajectory which approaches a limit of $\infty$ as $t \to \infty$ should be considered to have a fixed point at $\infty$.

Sensitive dependence on initial conditions means that nearby trajectories separate exponentially fast; \PMlinkname{i.e.}{Ie}, the system has a positive Liapunov exponent.

Strogatz notes that he favors additional constraints on the aperiodic long-term \PMlinkescapetext{behavior}, but leaves \PMlinkname{open}{OpenQuestion} what form they may take. He suggests two alternatives to fulfill this:

\begin{enumerate}
\item Requiring that there exists an open set of initial conditions having aperiodic trajectories, or
\item If one picks a random initial condition $x(0)$ then there must be a nonzero chance of the associated trajectory $x(t)$ being aperiodic.
\end{enumerate}

\subsection{Further reading}
\begin{enumerate}
\item  B. Codenotti and Luciano Margara. Chaos in Mathematics, Physics, and Computer Science: Similarities and Dissimilarities. http://pespmc1.vub.ac.be/Einmag\_Abstr/BCodenotti.html
\end{enumerate}

\subsection{References}
\begin{enumerate}
\item Steven H. Strogatz, "Nonlinear Dynamics and Chaos". Westview Press, 1994.
\end{enumerate}
%%%%%
%%%%%
\end{document}
