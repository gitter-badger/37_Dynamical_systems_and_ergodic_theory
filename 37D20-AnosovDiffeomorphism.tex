\documentclass[12pt]{article}
\usepackage{pmmeta}
\pmcanonicalname{AnosovDiffeomorphism}
\pmcreated{2013-03-22 13:47:43}
\pmmodified{2013-03-22 13:47:43}
\pmowner{Koro}{127}
\pmmodifier{Koro}{127}
\pmtitle{Anosov diffeomorphism}
\pmrecord{9}{34511}
\pmprivacy{1}
\pmauthor{Koro}{127}
\pmtype{Definition}
\pmcomment{trigger rebuild}
\pmclassification{msc}{37D20}
\pmdefines{Anosov flow}

% this is the default PlanetMath preamble.  as your knowledge
% of TeX increases, you will probably want to edit this, but
% it should be fine as is for beginners.

% almost certainly you want these
\usepackage{amssymb}
\usepackage{amsmath}
\usepackage{amsfonts}
\usepackage{mathrsfs}

% used for TeXing text within eps files
%\usepackage{psfrag}
% need this for including graphics (\includegraphics)
%\usepackage{graphicx}
% for neatly defining theorems and propositions
%\usepackage{amsthm}
% making logically defined graphics
%%%\usepackage{xypic}

% there are many more packages, add them here as you need them

% define commands here
\newcommand{\C}{\mathbb{C}}
\newcommand{\R}{\mathbb{R}}
\newcommand{\N}{\mathbb{N}}
\newcommand{\Z}{\mathbb{Z}}
\newcommand{\Per}{\operatorname{Per}}
\begin{document}
If $M$ is a compact smooth manifold, a diffeomorphism $f\colon M\to M$ (or a flow $\phi\colon\R\times M\to M$) such that the whole space $M$ is an hyperbolic set for $f$ (or $\phi$) is called an \emph{Anosov diffeomorphism} (or flow). 

Anosov diffeomorphisms were introduced by D.V. Anosov, who proved that they are $\mathcal{C}^1$-structurally stable.

Not every manifold admits an Anosov diffeomorphism; for example, there are no such diffeomorphisms on the sphere $S^n$. The simplest examples of compact manifolds admiting them are the tori $\mathbb{T}^n$: they admit the so called linear Anosov diffeomorphisms, which are isomorphisms of $\mathbb{T}^n$ having no eigenvalue of modulus $1$. It was proved that any other Anosov diffeomorphism in $\mathbb{T}^n$ is topologically conjugate to one of this kind. 

It is not known which manifolds support Anosov diffeomorphisms. The only known examples of are nilmanifolds and infranilmanifolds, and it is conjectured that these are the only ones. Anosov flows are more abundant; for example, if $M$ is a Riemannian manifold of negative sectional curvature, then its geodesic flow is an Anosov flow. 

Another famous conjecture is that the nonwandering set of any Anosov diffeomorphism is the whole manifold $M$. This is known to be true for linear Anosov diffeomorphisms (and hence for any Anosov diffeomorphism in a torus). For Anosov flows, there are examples where the nonwandering set is a proper subset of $M$.
%%%%%
%%%%%
\end{document}
