\documentclass[12pt]{article}
\usepackage{pmmeta}
\pmcanonicalname{FeigenbaumConstant}
\pmcreated{2013-03-22 12:34:25}
\pmmodified{2013-03-22 12:34:25}
\pmowner{yark}{2760}
\pmmodifier{yark}{2760}
\pmtitle{Feigenbaum constant}
\pmrecord{11}{32822}
\pmprivacy{1}
\pmauthor{yark}{2760}
\pmtype{Definition}
\pmcomment{trigger rebuild}
\pmclassification{msc}{37G15}
\pmsynonym{Feigenbaum delta constant}{FeigenbaumConstant}
\pmsynonym{Feigenbaum bifurcation velocity constant}{FeigenbaumConstant}
\pmsynonym{Feigenbaum bifurcation velocity}{FeigenbaumConstant}
\pmsynonym{Feigenbaum number}{FeigenbaumConstant}
\pmsynonym{Feigenbaum's number}{FeigenbaumConstant}
\pmsynonym{Feigenbaum's constant}{FeigenbaumConstant}

\endmetadata

\usepackage{amssymb}
\usepackage{amsmath}
\usepackage{amsfonts}
\usepackage{graphicx}

\begin{document}
\PMlinkescapeword{even}
\PMlinkescapephrase{generated by}
\PMlinkescapeword{structure}
\PMlinkescapeword{theory}
\PMlinkescapeword{tree}
\PMlinkescapeword{types}

The \emph{Feigenbaum delta constant} has the value

\[
  \delta = 4.669201609102990671853203820466 \ldots
\]

It governs the structure and behavior of many types of dynamical systems.
It was discovered in the 1970s by
\PMlinkexternal{Mitchell Feigenbaum}{http://www-groups.dcs.st-and.ac.uk/~history/Mathematicians/Feigenbaum.html},
while studying the logistic map

\[
  y'= \mu \cdot y (1- y),
\]

which produces the Feigenbaum tree:

\begin{center}
\includegraphics[scale=.8]{feigen_bifurcate.eps}

{\tiny Generated by GNU Octave and GNUPlot.}
\end{center}

If the bifurcations in this tree (first few shown as dotted blue lines)
are at points $b_1, b_2, b_3, \ldots$, then

\[
  \lim_{n\rightarrow \infty} \frac{b_{n}-b_{n-1}}{b_{n+1}-b_{n}} = \delta.
\]

That is, the ratio of the intervals between the bifurcation points
approaches Feigenbaum's constant.

However, this is only the beginning.
Feigenbaum discovered that this constant
arose in \emph{any} dynamical system
that approaches chaotic behavior via period-doubling bifurcation,
and has a single quadratic maximum.
So in some sense, Feigenbaum's constant
is a universal constant of chaos theory.

Feigenbaum's constant appears in problems of fluid-flow turbulence,
electronic oscillators, chemical reactions, and even the Mandelbrot set
(the ``budding'' of the Mandelbrot set along the negative real axis
occurs at intervals determined by Feigenbaum's constant).

\begin{thebibliography}{3}
\bibitem{sloane} \PMlinkexternal{A006890}{http://www.research.att.com/~njas/sequences/A006890}, ``Decimal expansion of Feigenbaum bifurcation velocity'', in the \PMlinkname{On-Line Encyclopedia of Integer Sequences}{OnLineEncyclopediaOfIntegerSequences}
\bibitem{BI} ``Bifurcations'': \PMlinkexternal{http://mcasco.com/bifurcat.html}{http://mcasco.com/bifurcat.html}
\end{thebibliography}
%%%%%
%%%%%
\end{document}
