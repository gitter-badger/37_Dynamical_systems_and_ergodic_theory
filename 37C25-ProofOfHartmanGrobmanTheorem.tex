\documentclass[12pt]{article}
\usepackage{pmmeta}
\pmcanonicalname{ProofOfHartmanGrobmanTheorem}
\pmcreated{2013-03-22 14:25:29}
\pmmodified{2013-03-22 14:25:29}
\pmowner{Koro}{127}
\pmmodifier{Koro}{127}
\pmtitle{proof of Hartman-Grobman theorem}
\pmrecord{7}{35933}
\pmprivacy{1}
\pmauthor{Koro}{127}
\pmtype{Proof}
\pmcomment{trigger rebuild}
\pmclassification{msc}{37C25}

\endmetadata

% this is the default PlanetMath preamble.  as your knowledge
% of TeX increases, you will probably want to edit this, but
% it should be fine as is for beginners.

% almost certainly you want these
\usepackage{amssymb}
\usepackage{amsmath}
\usepackage{amsfonts}
\usepackage{mathrsfs}

% used for TeXing text within eps files
%\usepackage{psfrag}
% need this for including graphics (\includegraphics)
%\usepackage{graphicx}
% for neatly defining theorems and propositions
\usepackage{amsthm}
% making logically defined graphics
%%%\usepackage{xypic}

% there are many more packages, add them here as you need them

% define commands here
\newcommand{\lip}{\operatorname{Lip}}
\newcommand{\C}{\mathbb{C}}
\newcommand{\R}{\mathbb{R}}
\newcommand{\N}{\mathbb{N}}
\newcommand{\Z}{\mathbb{Z}}
\newcommand{\Per}{\operatorname{Per}}
\newtheorem{proposition}{Proposition}
\newtheorem{lemma}{Lemma}
\renewcommand{\phi}{\varphi}
\renewcommand{\epsilon}{\varepsilon}
\begin{document}
\begin{lemma}
Let $A\colon E\to E$ be an hyperbolic isomorphism, and let $\phi$ and $\psi$
be $\epsilon$-Lipschitz maps from $E$ to itself such that $\phi(0)=\psi(0)=0$.
If $\epsilon$ is sufficiently small, then $A+\phi$ and $A+\psi$ are topologically conjugate.
\end{lemma}

Since $A$ is hyperbolic, we have $E=E^s \oplus E^u$ and there is $\lambda < 1$
(possibly changing the norm of $E$ by an equivalent box-type one), called the skewness of $A$,
 such that $$\|A|_{E^s}\|<\lambda, \quad \|A^{-1}|_{E^u}\|<\lambda$$ and
$$\|x\| = \max\{\|x_s\|,\|x_u\|\}.$$

Let us denote by $(\tilde E,\|\cdot\|_0)$ the Banach space of all bounded, continuous maps from $E$
to itself, with the norm of the supremum induced by the norm of $E$.
The operator $A$ induces a linear operator $\tilde A:\tilde E\to \tilde E$ defined by
$(\tilde Au)(x) = A(u(x))$, which is also hyperbolic. In fact, letting
$\tilde E^i$ be the set of all maps $u\colon \tilde E\to \tilde E$ whose range is contained
in $E^i$ (for $i= s,\,u$) we have that $\tilde E = \tilde E^s\oplus \tilde E^u$ is a hyperbolic splitting
for $\tilde A$ with the same skewness as $A$.

From now on we denote the projection of $x$ to $E^i$ by $x_i$, and
the restriction $A|_{E_i}\colon E^i\to E^i$ by $A_i$ ($i=s,\,u$).

We will try to find a conjugation of the form $I+u$ where $u\in \tilde E$.

\begin{proposition} There exists $\epsilon>0$ such that if $\phi$ and $\psi$ are $\epsilon$-Lipschitz,
then there is a unique $u\in \tilde E$ such that
$$(I+u)(A+\phi) = (A+\psi)(I+u).$$
\end{proposition}
\begin{proof}
We want to find $u$ such that
$$A+\phi + u(A+\phi) = A + Au + \psi(I+u)$$
which is the same as
$$\phi + u(A+\phi) = Au + \psi(I+u).$$
This can be rewriten as
\begin{align*}
u_u &=  A_u^{-1}(u_u(A+\phi) + \phi_u - \psi_u(I+u))\\
u_s &= (A_su_s + \psi_s(I+u) - \phi_s)(A+\phi)^{-1},
\end{align*}
where we use the fact that by the Lipschitz inverse mapping theorem, if $\lip(\phi)<1/\lambda \leq \|A^{-1}\|^{-1}$
(where $\lambda$ is the skewness of $A$) then $A+\phi$ is invertible with Lipschitz inverse.

Now define $\Gamma:\tilde E\to \tilde E$ by
\begin{align*}
\Gamma_s(u) &= ( A_su_s + \psi_s(I+u) - \phi_s)(A+\phi)^{-1} \\
\Gamma_u(u) &= A_u^{-1}(u_u(A+\phi) + \phi_u - \psi_u(I+u))
\end{align*}

We assert that, if $\epsilon$ is small, $\Gamma$ is a contraction.
In fact,

\begin{align*}
\left\|\Gamma_s(u) - \Gamma_s(v)\right\|_0 &= \left\|\left(A_s(u_s-v_s) + \psi_s(I+u) -
\psi_s(I+v)\right)(A+\phi)^{-1}\right\|_0 \\
&\leq \|\tilde A_s\|\cdot\|(u_s-v_s)(A+\phi)^{-1}\|_0 + \|(\psi_s(I+u) -
\psi_s(I+v))(A+\phi)^{-1}\|_0 \\
&\leq \lambda\|u_s-v_s\|_0 + \epsilon\|u-v\|_0\\
&\leq (\lambda+\epsilon) \|u-v\|_0
\end{align*}
and
\begin{align*}
\|\Gamma_u(u)-\Gamma_u(v)\|_0 &= \left\|A_u^{-1}(u_u(A+\phi) - v_u(A+\phi) - \psi_u(I+u) +
\psi_u(I+v))\right\|_0 \\
        &\leq \|\tilde A_u^{-1}\|\cdot\left( \|u_u(A+\phi)-v_u(A+\phi)\|_0 +
        \|\psi_u(I+u) - \psi_u(I+v)\|_0\right) \\
        &\leq \lambda \left( \|u_u - v_u\|_0 + \epsilon \|u - v\|_0 \right)\\
        &\leq \lambda(1 + \epsilon)\|u-v\|_0.
\end{align*}
Thus, if $\epsilon < \epsilon_0 \doteq \min\{\lambda, (1-\lambda)/\lambda\}$, $\Gamma$ has Lipschitz constant
smaller than $1$, so it is a contraction. Hence $u$ exists and is unique.
\end{proof}

\begin{proposition} The map $u$ from the previous proposition is a homeomorphism.
\end{proposition}
\begin{proof} Using the previous proposition with $\phi$ and $\psi$ switched, we get a unique
$v\in \tilde E$ such that
$$(I+v)(A+\psi) = (A+\phi)(I+v).$$
It follows that
\begin{equation}\label{eq1}
(I+v)(I+u)(A+\phi) = (I+v)(A+\psi)(I+u) = (A+\phi)(I+v)(I+u).
\end{equation}
Also, the previous proposition with $\phi = \psi$ implies that
that there is a unique $w\in\tilde E$ such that
$$(I+w)(A+\phi) = (A+\phi)(I+w),$$
which obviously is $w=0$. But since $(I+v)(I+u) = I+(u+v+uv)$ and $u+v+uv\in \tilde E$,
(\ref{eq1}) implies that $w=u+v+uv$ is a solution of the above equation, so that $u+v+uv = 0$ and $(I+v)(I+u) = I$. In a similar way, we see that $(I+u)(I+v)=I$. Hence $I+u$ is invertible, with continuous inverse.
\end{proof}

The two previous propositions prove the lemma.

\begin{proposition} If $U$ is an open neighborhood of $0$ and $f\colon U\to E$
is a $\mathcal{C^1}$ map with $f(0)=0$, then for every $\epsilon>0$ there is $\delta>0$ such that
$\phi\doteq f-Df(0)$ is $\epsilon$-Lipschitz in the ball $B(0,\delta)$.
\end{proposition}
\begin{proof} This is a direct consequence of the mean value inequality and the
fact that $D\phi$ is continuous and $D\phi(0)=0$.
\end{proof}

\begin{proposition} There is a constant $k$ such that if $\phi\colon \overline B(0,r)\to E$ is an $\epsilon$-Lipschitz map, then 
there is a $k\epsilon$-Lipschitz map $\tilde \phi\colon E\to E$
which coincides with $\phi$ in $B(0,r/2)$.
\end{proposition}
\begin{proof}
Let $\eta\colon \R \to \R$ be a $\mathcal{C}^\infty$ bump function: an infinitely
differentiable  map such that
$\eta(x) = 1$ for $x<1/2$ and $\eta(x) = 0$ for $x>1$, with derivative bounded by $M$ and $|\eta(x)\|\leq 1$ for all $x\in \R$.
Now define $\tilde \phi(x) = \phi(x)\eta(\|x\|/r)$ (when $\phi(x)$ is not
defined, we assume that it is zero).
If $x$ and $y$ are both in $B(0,r)$ then we have
\begin{align*}
\|\tilde\phi(x)-\tilde\phi(y)\| &= \big\|\phi(x)\eta(\|x\|/r) -
\phi(y)\eta(\|y\|/r)\big\| \\
&\leq \big\|(\phi(x)-\phi(y))\eta(\|x\|/r)\big\| +
\big\|\phi(y)(\eta(\|x\|/r)-\eta(\|y\|/r))\big\|\\
&\leq \epsilon\|x-y\| +
\|\phi(y)-\phi(0)\|\cdot\big\|\eta(\|x\|/r)-\eta(\|y\|/r)\big\|\\
&\leq \epsilon\|x-y\| + \epsilon\|y\|(M\|x-y\|/r)\\
&\leq (M+1)\epsilon\|x-y\|;
\end{align*}
if $x$ is in $B(0,r)$ and $y$ is not, then
$$\|\tilde\phi(x)-\tilde\phi(y)\| = \|\tilde\phi(x) - \tilde\phi(y^*)\|,$$
where $y^*$ is defined as $x+\tau(y-x)$ with
$$\tau = \sup\{t: x+t(y-x)\in E\setminus B(0,r)\}$$
This is true because $\tilde\phi(y^*)=0$. Also, $\|x-y^*\| = \tau\|x - y\|
\leq \|x-y\|$; hence
$$\|\tilde\phi(x)-\tilde\phi(y)\| = \|\tilde\phi(x) - \tilde\phi(y^*)\|\leq
(M+1)\epsilon\|x-y^*\| \leq (M+1)\epsilon\|x-y\|.$$
Finally, if both $x$ and $y$ are outside $B(0,r)$, then
$\|\tilde\phi(x)-\tilde\phi(y)\| = 0 \leq (M+1)\|x-y\|$. Letting $k=M+1$ we
get the desired result.
\end{proof}

\textbf{Proof of the theorem.}
Taking the particular $\psi = 0$ in the lemma, we observe that
there is $\epsilon>0$ such that for any $\epsilon$-Lipschitz map $\phi$,
$Df(0)$ is conjugate to $\phi + Df(0)$.
 Choose $\delta$ such that $f-Df(0)$ is $\epsilon/k$-Lipschitz in $B(0,2\delta)$.
Let $\tilde\phi$ be the $\epsilon$-Lipschitz extension of $f-Df(0)$ to
$B(0,\delta)$ obtained from the previous proposition. We have that
$Df(0)+\tilde\phi$ is conjugate to $Df(0)$. But for $x\in B(0,\delta)$ we have
$Df(0)+\tilde\phi = f$, so that $f$ is locally conjugate to $Df(0)$.
\end{document}
%%%%%
%%%%%
\end{document}
