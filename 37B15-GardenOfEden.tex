\documentclass[12pt]{article}
\usepackage{pmmeta}
\pmcanonicalname{GardenOfEden}
\pmcreated{2013-03-22 19:22:01}
\pmmodified{2013-03-22 19:22:01}
\pmowner{Ziosilvio}{18733}
\pmmodifier{Ziosilvio}{18733}
\pmtitle{Garden of Eden}
\pmrecord{6}{42320}
\pmprivacy{1}
\pmauthor{Ziosilvio}{18733}
\pmtype{Definition}
\pmcomment{trigger rebuild}
\pmclassification{msc}{37B15}
\pmclassification{msc}{68Q80}
\pmdefines{orphan pattern (cellular automaton)}
\pmdefines{orphan pattern principle}

\endmetadata

% this is the default PlanetMath preamble.  as your knowledge
% of TeX increases, you will probably want to edit this, but
% it should be fine as is for beginners.

% almost certainly you want these
\usepackage{amssymb}
\usepackage{amsmath}
\usepackage{amsfonts}

% used for TeXing text within eps files
%\usepackage{psfrag}
% need this for including graphics (\includegraphics)
%\usepackage{graphicx}
% for neatly defining theorems and propositions
%\usepackage{amsthm}
% making logically defined graphics
%%%\usepackage{xypic}

% there are many more packages, add them here as you need them

% define commands here

\begin{document}
\newcommand{\Acal}{\ensuremath{\mathcal{A}}}
\newcommand{\goe}{\textsc{GoE}}
\newcommand{\Neigh}{\ensuremath{\mathcal{N}}}
\newcommand{\restrict}[2]{\ensuremath{\left.{#1}\right|_{#2}}}

\newtheorem{lemma}{Lemma}

A \emph{Garden of Eden} (briefly, \goe)
for a cellular automaton
\begin{math}
\Acal = \langle Q, \Neigh, f \rangle
\end{math}
on a group $G$
is a configuration $c \in Q^G$
which is not in the image of the global function $F_\Acal$ of $\Acal$.

In other words, a Garden of Eden is a global situation
which can be started from, but never returned to.

The finitary counterpart of a \goe\ configuration
is an \emph{orphan pattern}:
a pattern which cannot be obtained
by synchronous application of the local function $f$.

Of course,
any cellular automaton with an orphan pattern
also has a \goe\ configuration.
\begin{lemma}[Orphan pattern principle] \label{lem:opp}
If a cellular automaton with finite set of states $Q$
has a \goe\ configuration,
then it also has an orphan pattern.
\end{lemma}
\textit{Proof.}
First, suppose that $G$ is countable.
Let
\begin{math}
\Acal = \langle Q, \Neigh, f \rangle
\end{math}
be a cellular automaton with no orphan pattern.
Let $c : G \to Q$ be a configuration:
we will prove that there is some $e : G \to Q$
such that $F_\Acal(e) = c$.

Let
\begin{math}
G = \{g_n\}_{n \geq 0}
\end{math}
be an enumeration of $G$:
put $E_n = \{g_i \mid i \leq n\}$
and let $p_n : E_n \to Q$ be defined as
\begin{math}
p_n = \restrict{c}{E_n}.
\end{math}
By hypothesis, none of the $p_n$'s is an orphan,
so there is a sequence of configurations
$c_n : G \to Q$
satisfying
\begin{math}
\restrict{F_\Acal(c_n)}{E_n} = p_n = \restrict{c}{E_n}.
\end{math}
It is easy to see that
\begin{math}
\lim_{k \to \infty} F_\Acal(c_{n_k}) = c.
\end{math}
But if $Q$ is finite,
then $Q^G$ is compact by Tychonoff's theorem,
so there exista a subsequence
\begin{math}
\{ c_{n_k} \}_{k \geq 0}
\end{math}
and a configuration $e : G \to Q$ satisfying
\begin{math}
\lim_{k \to \infty} c_{n_k} = e:
\end{math}
Since $F_\Acal$ is continuous in the product topology, $F(e) = c$.

Let now $G$ be arbitrary.
Let $H$ be the subgroup generated by the neighborhood index $\Neigh$:
since $\Neigh$ is finite, $H$ is countable.
Let $J$ be a set of representatives of the left cosets of $H$ in $G$,
so that
\begin{math}
G = \bigsqcup_{j \in  J} jH.
\end{math}
(Observe that we do \emph{not} require that $H$ is normal in $G$.)
Call $\Acal_H$ the cellular automaton \emph{on $H$}
that has the same local description
(set of states, neighborhood index, local function) as $\Acal$.
Let $c : G \to Q$ be a Garden of Eden configuration for $\Acal$:
then at least one of the configurations
\begin{math}
c_j(h) = c(jh)
\end{math}
must be a Garden of Eden for $\Acal_H$.
By the discussion above, $\Acal_H$ must have an orphan pattern,
which is also an orphan pattern for $\Acal$.
\hfill $\Box$

%%%%%
%%%%%
\end{document}
