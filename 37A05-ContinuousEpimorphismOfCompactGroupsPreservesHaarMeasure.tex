\documentclass[12pt]{article}
\usepackage{pmmeta}
\pmcanonicalname{ContinuousEpimorphismOfCompactGroupsPreservesHaarMeasure}
\pmcreated{2013-03-22 17:59:06}
\pmmodified{2013-03-22 17:59:06}
\pmowner{asteroid}{17536}
\pmmodifier{asteroid}{17536}
\pmtitle{continuous epimorphism of  compact groups preserves Haar measure}
\pmrecord{9}{40495}
\pmprivacy{1}
\pmauthor{asteroid}{17536}
\pmtype{Theorem}
\pmcomment{trigger rebuild}
\pmclassification{msc}{37A05}
\pmclassification{msc}{28C10}
\pmclassification{msc}{22C05}

\endmetadata

% this is the default PlanetMath preamble.  as your knowledge
% of TeX increases, you will probably want to edit this, but
% it should be fine as is for beginners.

% almost certainly you want these
\usepackage{amssymb}
\usepackage{amsmath}
\usepackage{amsfonts}

% used for TeXing text within eps files
%\usepackage{psfrag}
% need this for including graphics (\includegraphics)
%\usepackage{graphicx}
% for neatly defining theorems and propositions
%\usepackage{amsthm}
% making logically defined graphics
%%%\usepackage{xypic}

% there are many more packages, add them here as you need them

% define commands here

\begin{document}
\PMlinkescapephrase{right}
\PMlinkescapephrase{preserves}

{\bf Theorem -} Let $G, H$ be compact Hausdorff topological groups. If $\phi:G \longrightarrow H$ is a continuous surjective homomorphism, then $\phi$ is a measure preserving transformation, in the sense that it preserves the normalized Haar measure.

$\,$

{\bf \emph{\PMlinkescapetext{Proof}:}} Let $\mu$ be the Haar measure in $G$ (normalized, i.e. $\mu (G) = 1$). Let $\nu$ be defined for measurable subsets $E$ of $H$ by
\begin{displaymath}
\nu(E)=\mu(\phi^{-1}(E))
\end{displaymath}
It is easy to see that $\nu$ defines a measure in $H$. Let us now see that $\nu$ is invariant under right translations. For every $s \in G$ and every measurable subset $E \subset H$ we have that
\begin{align}
\phi^{-1}(\phi(s)E)=s\phi^{-1}(E)
\end{align}
The inclusion $\supseteq$ is obvious. To prove the other inclusion notice that if $z \in \phi^{-1}(\phi(s)E)$ then $\phi(z) = \phi(s)t$ for some $t \in E$. Hence, $\phi(s^{-1}z)=t$, i.e $s^{-1}z \in \phi^{-1}(E)$. It now follows that $z=s(s^{-1}z) \in s\phi^{-1}(E)$.

Thus, equality (1) and the fact that $\mu$ is a Haar measure imply that
\begin{displaymath}
\nu(\phi(s)E)= \mu \big(\phi^{-1}(\phi(s)E)\big)=\mu(s\phi^{-1}(E)) = \mu(\phi^{-1}(E)) = \nu(E)
\end{displaymath}

Since $\phi$ is surjective it follows that $\nu$ is right invariant. It is not difficult to see that $\nu$ is regular, finite on compact sets and $\nu(H) = 1$. Hence, $\nu$ is the normalized Haar measure in $H$ and, by definition, we have that
\begin{displaymath}
\nu(E)=\mu(\phi^{-1}(E))
\end{displaymath}
Thus, $\phi$ preserves the Haar measure. $\square$
%%%%%
%%%%%
\end{document}
