\documentclass[12pt]{article}
\usepackage{pmmeta}
\pmcanonicalname{TopologicallyTransitive}
\pmcreated{2013-03-22 13:41:05}
\pmmodified{2013-03-22 13:41:05}
\pmowner{Koro}{127}
\pmmodifier{Koro}{127}
\pmtitle{topologically transitive}
\pmrecord{5}{34354}
\pmprivacy{1}
\pmauthor{Koro}{127}
\pmtype{Definition}
\pmcomment{trigger rebuild}
\pmclassification{msc}{37B99}
\pmclassification{msc}{54H20}
\pmdefines{topologically mixing}
\pmdefines{topological mixing}

% this is the default PlanetMath preamble.  as your knowledge
% of TeX increases, you will probably want to edit this, but
% it should be fine as is for beginners.

% almost certainly you want these
\usepackage{amssymb}
\usepackage{amsmath}
\usepackage{amsfonts}
\usepackage{mathrsfs}

% used for TeXing text within eps files
%\usepackage{psfrag}
% need this for including graphics (\includegraphics)
%\usepackage{graphicx}
% for neatly defining theorems and propositions
%\usepackage{amsthm}
% making logically defined graphics
%%%\usepackage{xypic}

% there are many more packages, add them here as you need them

% define commands here
\newcommand{\C}{\mathbb{C}}
\newcommand{\R}{\mathbb{R}}
\newcommand{\N}{\mathbb{N}}
\newcommand{\Z}{\mathbb{Z}}
\newcommand{\Per}{\operatorname{Per}}
\begin{document}
A continuous surjection $f$ on a topological space $X$ to itself 
is \emph{topologically transitive} if for every
pair of open sets $U$ and $V$ in $X$ there is an integer $n>0$ 
such that $f^n(U)\cap V\neq \emptyset$, where $f^n$ denotes the $n$-th iterate of $f$.

If for every pair of open sets $U$ and $V$ there is an integer $N$ such that
$f^n(U)\cap V\neq \emptyset$ for each $n>N$, we say that $f$ is \emph{topologically mixing}.

If $X$ is a compact metric space, then $f$ is topologically transitive if and only if there exists a point $x\in X$ with a dense orbit, i.e. such that $\mathcal{O}(x,f)=\{f^n(x): n\in \N\}$ is dense in $X$.
%%%%%
%%%%%
\end{document}
