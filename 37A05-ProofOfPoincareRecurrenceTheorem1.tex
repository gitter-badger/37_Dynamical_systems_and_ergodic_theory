\documentclass[12pt]{article}
\usepackage{pmmeta}
\pmcanonicalname{ProofOfPoincareRecurrenceTheorem1}
\pmcreated{2013-03-22 14:29:56}
\pmmodified{2013-03-22 14:29:56}
\pmowner{Koro}{127}
\pmmodifier{Koro}{127}
\pmtitle{proof of Poincar\'e recurrence theorem 1}
\pmrecord{5}{36035}
\pmprivacy{1}
\pmauthor{Koro}{127}
\pmtype{Proof}
\pmcomment{trigger rebuild}
\pmclassification{msc}{37A05}
\pmclassification{msc}{37B20}

% this is the default PlanetMath preamble.  as your knowledge
% of TeX increases, you will probably want to edit this, but
% it should be fine as is for beginners.

% almost certainly you want these
\usepackage{amssymb}
\usepackage{amsmath}
\usepackage{amsfonts}
\usepackage{mathrsfs}

% used for TeXing text within eps files
%\usepackage{psfrag}
% need this for including graphics (\includegraphics)
%\usepackage{graphicx}
% for neatly defining theorems and propositions
%\usepackage{amsthm}
% making logically defined graphics
%%%\usepackage{xypic}

% there are many more packages, add them here as you need them

% define commands here
\newcommand{\C}{\mathbb{C}}
\newcommand{\R}{\mathbb{R}}
\newcommand{\N}{\mathbb{N}}
\newcommand{\Z}{\mathbb{Z}}
\newcommand{\Per}{\operatorname{Per}}
\begin{document}
Let $A_n= \cup_{k=n}^\infty f^{-k}E$. Clearly, $E\subset A_0$ and
$A_i\subset A_j$ when $j\leq i$. Also, $A_i =
f^{j-i}A_j$, so that $\mu(A_i)=\mu(A_j)$ for all $i,j\geq 0$, by
the $f$-invariance of $\mu$.
Now for any $n>0$ we have $E-A_n\subset A_0-A_n$, so that
$$\mu(E-A_n) \leq \mu(A_0-A_n) = \mu(A_0) - \mu(A_n) = 0.$$
Hence $\mu(E-A_n)=0$ for all $n>0$, so that
$\mu(E-\cap_{n=1}^\infty A_n) = \mu(\cup_{n=1}^\infty E-A_n)=0$.
But $E-\cap_{n=1}^\infty A_n$ is precisely the set of those $x\in E$ such
that for some $n$ and for all $k>n$ we have $f^k(x)\notin E$. $\Box$
%%%%%
%%%%%
\end{document}
