\documentclass[12pt]{article}
\usepackage{pmmeta}
\pmcanonicalname{Gammaequivariant}
\pmcreated{2013-03-22 13:53:20}
\pmmodified{2013-03-22 13:53:20}
\pmowner{mathcam}{2727}
\pmmodifier{mathcam}{2727}
\pmtitle{$\Gamma$-equivariant}
\pmrecord{7}{34634}
\pmprivacy{1}
\pmauthor{mathcam}{2727}
\pmtype{Definition}
\pmcomment{trigger rebuild}
\pmclassification{msc}{37C80}
\pmclassification{msc}{22-00}

% this is the default PlanetMath preamble.  as your knowledge
% of TeX increases, you will probably want to edit this, but
% it should be fine as is for beginners.

% almost certainly you want these
\usepackage{amssymb}
\usepackage{amsmath}
\usepackage{amsfonts}

% used for TeXing text within eps files
%\usepackage{psfrag}
% need this for including graphics (\includegraphics)
%\usepackage{graphicx}
% for neatly defining theorems and propositions
%\usepackage{amsthm}
% making logically defined graphics
%%%\usepackage{xypic} 

% there are many more packages, add them here as you need them

% define commands here
\begin{document}
Let $\Gamma$ be a compact Lie group acting linearly on $V$ and let $g$ be a mapping defined as $g\colon V \to V$.  Then $g$ is \emph{$\Gamma$-equivariant} if $$g(\gamma v)=\gamma g(v)$$
for all $\gamma \in \Gamma$, and all $v \in V$.\\
Therefore if $g$ commutes with $\Gamma$ then $g$ is $\Gamma$-equivariant. 

\cite{1}
\begin{thebibliography}{1}
\bibitem[GSS]{1} Golubitsky, Martin. Stewart, Ian. Schaeffer, G. David.: Singularities and Groups in Bifurcation Theory \textit{(Volume II)}. Springer-Verlag, New York, 1988.
\end{thebibliography}
%%%%%
%%%%%
\end{document}
