\documentclass[12pt]{article}
\usepackage{pmmeta}
\pmcanonicalname{HyperbolicIsomorphism}
\pmcreated{2013-03-22 13:39:34}
\pmmodified{2013-03-22 13:39:34}
\pmowner{Koro}{127}
\pmmodifier{Koro}{127}
\pmtitle{hyperbolic isomorphism}
\pmrecord{10}{34315}
\pmprivacy{1}
\pmauthor{Koro}{127}
\pmtype{Definition}
\pmcomment{trigger rebuild}
\pmclassification{msc}{37D05}
\pmclassification{msc}{46B03}
\pmsynonym{linear hyperbolic isomorphism}{HyperbolicIsomorphism}

% this is the default PlanetMath preamble.  as your knowledge
% of TeX increases, you will probably want to edit this, but
% it should be fine as is for beginners.

% almost certainly you want these
\usepackage{amssymb}
\usepackage{amsmath}
\usepackage{amsfonts}
\usepackage{mathrsfs}

% used for TeXing text within eps files
%\usepackage{psfrag}
% need this for including graphics (\includegraphics)
%\usepackage{graphicx}
% for neatly defining theorems and propositions
%\usepackage{amsthm}
% making logically defined graphics
%%%\usepackage{xypic}

% there are many more packages, add them here as you need them

% define commands here
\newcommand{\C}{\mathbb{C}}
\newcommand{\R}{\mathbb{R}}
\newcommand{\N}{\mathbb{N}}
\newcommand{\Z}{\mathbb{Z}}
\begin{document}
Let $X$ be a Banach space and $T:X\to X$ a continuous linear isomorphism. We say that $T$ is an \emph{hyperbolic isomorphism} if its spectrum is disjoint with the unit circle, i.e. $\sigma(T)\cap \{z\in \C:|z|=1\}=\emptyset$.

If this is the case, by the spectral theorem there is a 
splitting of $X$ into two invariant subspaces, $X=E^s\oplus E^u$ (and therefore, a corresponding splitting of $T$ into two operators $T^s:E^s\to E^s$ and $T_u:E^u\to E^u$, i.e. $T=T_s\oplus
T_u$), such that $\sigma(T_s) = \sigma(T)\cap \{z:|z|<1\}$ and $\sigma(T_u)=\sigma(T)\cap\{z:|z|>1\}$. Also, for any $\lambda$ greater than the spectral radius of both $T_s$ and $T_u^{-1}$ there exists an equivalent (box-type) norm $\|\cdot\|_1$ such that 
$$\|T_s\|_1 < \lambda \textnormal{ and } \|T_u^{-1}\|_1 < \lambda$$
and $$\|x\|_1 = \max\{\|x_u\|_1,\|x_s\|_1\}.$$
In particular, $\lambda$ can be chosen smaller than $1$, so that $T_s$ and $T_u^{-1}$ are contractions.
%%%%%
%%%%%
\end{document}
