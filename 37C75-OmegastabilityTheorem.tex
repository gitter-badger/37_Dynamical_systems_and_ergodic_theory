\documentclass[12pt]{article}
\usepackage{pmmeta}
\pmcanonicalname{OmegastabilityTheorem}
\pmcreated{2013-03-22 14:30:55}
\pmmodified{2013-03-22 14:30:55}
\pmowner{Koro}{127}
\pmmodifier{Koro}{127}
\pmtitle{$\Omega$-stability theorem}
\pmrecord{8}{36055}
\pmprivacy{1}
\pmauthor{Koro}{127}
\pmtype{Theorem}
\pmcomment{trigger rebuild}
\pmclassification{msc}{37C75}
\pmsynonym{omega-stability theorem}{OmegastabilityTheorem}
\pmsynonym{Smale's $\Omega$-stability theorem}{OmegastabilityTheorem}
\pmdefines{$\Omega$-stable}
\pmdefines{omega-stable}
\pmdefines{$\Omega$-stability}
\pmdefines{omega-stability}

\endmetadata

% this is the default PlanetMath preamble.  as your knowledge
% of TeX increases, you will probably want to edit this, but
% it should be fine as is for beginners.

% almost certainly you want these
\usepackage{amssymb}
\usepackage{amsmath}
\usepackage{amsfonts}
\usepackage{mathrsfs}

% used for TeXing text within eps files
%\usepackage{psfrag}
% need this for including graphics (\includegraphics)
%\usepackage{graphicx}
% for neatly defining theorems and propositions
%\usepackage{amsthm}
% making logically defined graphics
%%%\usepackage{xypic}

% there are many more packages, add them here as you need them

% define commands here
\newcommand{\C}{\mathbb{C}}
\newcommand{\R}{\mathbb{R}}
\newcommand{\N}{\mathbb{N}}
\newcommand{\Z}{\mathbb{Z}}
\newcommand{\Per}{\operatorname{Per}}
\begin{document}
\PMlinkescapeword{satisfies}

Let $M$ be a differentiable manifold and let $f\colon M\to M$ be a $\mathcal{C}^k$ diffeomorphism. We say that $f$ is $\mathcal{C}^k$-$\Omega$-stable, if there is a neighborhood $\mathcal{U}$ of $f$ in the $\mathcal{C}^k$ topology of $\operatorname{Diff}^k(M)$ such that for any $g\in \mathcal{U}$, $f|_{\Omega(f)}$ is topologically conjugate to $g|_{\Omega(g)}$.

$\Omega$-\textbf{stability theorem}. If $f$ is Axiom A and satisfies the no-cycles condition, then $f$ is $\mathcal{C}^1$-$\Omega$-stable.

\emph{Remark.} The reciprocal of this theorem is also true (the difficult part is showing that $\Omega$-stability implies Axiom A), but it is unknown whether $\mathcal{C}^k$-$\Omega$-stability implies Axiom A when $k>1$. This is known as the $\mathcal{C}^k$ \emph{$\Omega$-stability conjecture}.
%%%%%
%%%%%
\end{document}
