\documentclass[12pt]{article}
\usepackage{pmmeta}
\pmcanonicalname{SymbolicDynamics}
\pmcreated{2013-03-22 16:28:29}
\pmmodified{2013-03-22 16:28:29}
\pmowner{PrimeFan}{13766}
\pmmodifier{PrimeFan}{13766}
\pmtitle{symbolic dynamics}
\pmrecord{5}{38639}
\pmprivacy{1}
\pmauthor{PrimeFan}{13766}
\pmtype{Definition}
\pmcomment{trigger rebuild}
\pmclassification{msc}{37-00}

% this is the default PlanetMath preamble.  as your knowledge
% of TeX increases, you will probably want to edit this, but
% it should be fine as is for beginners.

% almost certainly you want these
\usepackage{amssymb}
\usepackage{amsmath}
\usepackage{amsfonts}

% used for TeXing text within eps files
%\usepackage{psfrag}
% need this for including graphics (\includegraphics)
%\usepackage{graphicx}
% for neatly defining theorems and propositions
%\usepackage{amsthm}
% making logically defined graphics
%%%\usepackage{xypic}

% there are many more packages, add them here as you need them

% define commands here

\begin{document}
In mathematics, {\em symbolic dynamics} is the practice of modelling a dynamical system by a space consisting of infinite sequences of abstract symbols, each symbol corresponding to a state of the system, and a shift operator corresponding to the dynamics. Symbolic dynamics was first introduced by Emil Artin in 1924, in the study of Artin billiards.

Symbolic dynamics originated as a method to study general dynamical systems, but its techniques and ideas have found significant applications in data storage and transmission, linear algebra, the motions of the planets and many other areas. The distinct feature in symbolic dynamics is that time is measured in discrete intervals. So at each time interval the system is in a particular state. Each state is associated with a symbol and the evolution of the system is described by an infinite sequence of symbols - represented effectively as strings. If the system states are not inherently discrete, then the state vector must be discretized, so as to get a coarse-grained description of the system.

% The below commented-out paragraph is not correct on PlanetMath, since
% we don't have an article on measure-preserving dynamical systems.+
%
%A formal definition of the symbolic dynamics of a dynamical system is given in %the article measure-preserving dynamical system. See also shift of finite type.

{\it This entry was adapted from the Wikipedia article \PMlinkexternal{Symbolic dynamics}{http://en.wikipedia.org/wiki/Symbolic_dynamics} as of December 19, 2006.}
%%%%%
%%%%%
\end{document}
