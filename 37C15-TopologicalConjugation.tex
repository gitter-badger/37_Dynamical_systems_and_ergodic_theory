\documentclass[12pt]{article}
\usepackage{pmmeta}
\pmcanonicalname{TopologicalConjugation}
\pmcreated{2013-03-22 13:41:02}
\pmmodified{2013-03-22 13:41:02}
\pmowner{Koro}{127}
\pmmodifier{Koro}{127}
\pmtitle{topological conjugation}
\pmrecord{14}{34353}
\pmprivacy{1}
\pmauthor{Koro}{127}
\pmtype{Definition}
\pmcomment{trigger rebuild}
\pmclassification{msc}{37C15}
\pmclassification{msc}{37B99}
\pmdefines{topologically conjugate}
\pmdefines{topological semiconjugation}
\pmdefines{topologically semiconjugate}
\pmdefines{topologically equivalent}
\pmdefines{topological equivalence}

% this is the default PlanetMath preamble.  as your knowledge
% of TeX increases, you will probably want to edit this, but
% it should be fine as is for beginners.

% almost certainly you want these
\usepackage{amssymb}
\usepackage{amsmath}
\usepackage{amsfonts}
\usepackage{mathrsfs}

% used for TeXing text within eps files
%\usepackage{psfrag}
% need this for including graphics (\includegraphics)
%\usepackage{graphicx}
% for neatly defining theorems and propositions
%\usepackage{amsthm}
% making logically defined graphics
%%%\usepackage{xypic}

% there are many more packages, add them here as you need them

% define commands here
\newcommand{\C}{\mathbb{C}}
\newcommand{\R}{\mathbb{R}}
\newcommand{\N}{\mathbb{N}}
\newcommand{\Z}{\mathbb{Z}}
\newcommand{\Per}{\operatorname{Per}}
\renewcommand{\emph}[1]{\textbf{#1}}
\begin{document}
Let $X$ and $Y$ be topological spaces, and let $f\colon X\to X$ and $g\colon Y\to Y$
be continuous functions. We say that $f$ is
\emph{topologically semiconjugate} to $g$, if there exists a continuous
surjection $h\colon Y\to X$ such that $fh=hg$. If $h$ is a homeomorphism,
then we say that $f$ and $g$ are \emph{topologically conjugate}, and we call
$h$ a \emph{topological conjugation} between $f$ and $g$.

Similarly, a flow $\varphi$ on $X$ is topologically semiconjugate to a flow $\psi$ on $Y$ if there is a continuous surjection $h\colon Y\to X$ such that 
$\varphi(h(y),t) = h\psi(y,t)$ for each $y\in Y$, $t\in \R$. If $h$ is a homeomorphism then $\psi$ and $\varphi$ are topologically conjugate.


\subsection{Remarks}

Topological conjugation defines an equivalence relation in the 
space of all continuous surjections of a topological space to itself, 
by declaring $f$ and $g$ to be related if they are topologically 
conjugate. This equivalence relation is very useful in the theory of
dynamical systems, since each class contains all functions which 
share the same dynamics from the topological viewpoint. In fact, orbits 
of $g$ are mapped to homeomorphic orbits of $f$ through the conjugation.
Writing $g = h^{-1}fh$ makes this fact evident: $g^n = h^{-1}f^nh$.
Speaking informally, topological conjugation is a ``change of coordinates'' in the topological sense.

However, the analogous definition for flows is somewhat restrictive. In fact, we are requiring the maps $\varphi(\cdot,t)$ and $\psi(\cdot,t)$ to be topologically conjugate for each $t$, which is requiring more than simply that  orbits of $\varphi$ be mapped to orbits of $\psi$ homeomorphically.
This motivates the definition of \emph{topological equivalence}, which also partitions the set of all flows in $X$ into classes of flows sharing the same dynamics, again from the topological viewpoint.

We say that $\psi$ and $\varphi$ are \emph{topologically equivalent}, if there is an homeomorphism $h:Y\to X$, mapping orbits of $\psi$ to orbits of $\varphi$ homeomorphically, and preserving orientation of the orbits. This means that:
\begin{enumerate}
\item $h(\mathcal{O}(y,\psi)) = \{h(\psi(y,t)): t\in\R\} = \{\varphi(h(y),t):t\in\R\}= \mathcal{O}(h(y),\varphi)$ for each $y\in Y$;
\item for each $y\in Y$, there is $\delta>0$ such that, if $0<|s|< t < \delta$, and if $s$ is such that $\varphi(h(y),s) = h(\psi(y,t))$, then $s>0$.
\end{enumerate}
%%%%%
%%%%%
\end{document}
