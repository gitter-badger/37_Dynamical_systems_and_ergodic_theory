\documentclass[12pt]{article}
\usepackage{pmmeta}
\pmcanonicalname{FrobeniusTheorem}
\pmcreated{2013-03-22 14:52:03}
\pmmodified{2013-03-22 14:52:03}
\pmowner{jirka}{4157}
\pmmodifier{jirka}{4157}
\pmtitle{Frobenius' theorem}
\pmrecord{9}{36543}
\pmprivacy{1}
\pmauthor{jirka}{4157}
\pmtype{Theorem}
\pmcomment{trigger rebuild}
\pmclassification{msc}{37C10}
\pmclassification{msc}{53-00}
\pmclassification{msc}{53B25}
\pmrelated{Distribution5}
\pmrelated{IntegralManifold}

\endmetadata

% this is the default PlanetMath preamble.  as your knowledge
% of TeX increases, you will probably want to edit this, but
% it should be fine as is for beginners.

% almost certainly you want these
\usepackage{amssymb}
\usepackage{amsmath}
\usepackage{amsfonts}

% used for TeXing text within eps files
%\usepackage{psfrag}
% need this for including graphics (\includegraphics)
%\usepackage{graphicx}
% for neatly defining theorems and propositions
\usepackage{amsthm}
% making logically defined graphics
%%%\usepackage{xypic}

% there are many more packages, add them here as you need them

% define commands here
\theoremstyle{theorem}
\newtheorem*{thm}{Theorem}
\newtheorem*{lemma}{Lemma}
\newtheorem*{conj}{Conjecture}
\newtheorem*{cor}{Corollary}
\newtheorem*{example}{Example}
\theoremstyle{definition}
\newtheorem*{defn}{Definition}
\begin{document}
\begin{thm}[Frobenius]
Let $M$ be a smooth manifold ($C^\infty$) and let $\Delta$ be a
distribution on $M$.  Then $\Delta$ is completely integrable if and only if
$\Delta$ is involutive.
\end{thm}

One direction in the proof is pretty easy since the tangent space
of an integral manifold is involutive, so sometimes the theorem is only
stated in one direction, that is:  If $\Delta$ is involutive then it is
completely integrable.

Another way to \PMlinkescapetext{state} the theorem is that if we have $n$ vector fields $\{X_k\}_{k=1}^n$ on a manifold $M$ such that they are linearly independent at every point of the manifold, and furthermore if for any $k,m$ we have
$[X_k,X_m] = \sum_{j=1}^n a_j X_j$ for some $C^\infty$ functions $a_j$, then for any point $x \in N$, there exists a germ of a submanifold $N \subset M$, through  $x$, such that $TN$ is spanned
by $\{X_k\}_{k=1}^n$.  Note that if we extend $N$ to all of $M$, it need not be
an embedded submanifold anymore, but just an immersed one.

For $n=1$ above, this is just the existence and uniqueness of solution of ordinary differential equations.

\begin{thebibliography}{9}
\bibitem{boothby}
William M.\@ Boothby.
{\em \PMlinkescapetext{An Introduction to Differentiable Manifolds and
Riemannian Geometry}},
Academic Press, San Diego, California, 2003.
\bibitem{wikipedia} Frobenius theorem at Wikipedia: \PMlinkexternal{http://en.wikipedia.org/wiki/Frobenius_theorem}{http://en.wikipedia.org/wiki/Frobenius_theorem}
\end{thebibliography}
%%%%%
%%%%%
\end{document}
