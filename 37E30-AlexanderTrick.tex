\documentclass[12pt]{article}
\usepackage{pmmeta}
\pmcanonicalname{AlexanderTrick}
\pmcreated{2013-03-22 15:53:38}
\pmmodified{2013-03-22 15:53:38}
\pmowner{juanman}{12619}
\pmmodifier{juanman}{12619}
\pmtitle{Alexander trick}
\pmrecord{7}{37895}
\pmprivacy{1}
\pmauthor{juanman}{12619}
\pmtype{Definition}
\pmcomment{trigger rebuild}
\pmclassification{msc}{37E30}
\pmclassification{msc}{57S05}
%\pmkeywords{map extension}
\pmrelated{Homeomorphism}

\endmetadata

% this is the default PlanetMath preamble.  as your knowledge
% of TeX increases, you will probably want to edit this, but
% it should be fine as is for beginners.

% almost certainly you want these
\usepackage{amssymb}
\usepackage{amsmath}
\usepackage{amsfonts}

% used for TeXing text within eps files
%\usepackage{psfrag}
% need this for including graphics (\includegraphics)
%\usepackage{graphicx}
% for neatly defining theorems and propositions
%\usepackage{amsthm}
% making logically defined graphics
%%%\usepackage{xypic}

% there are many more packages, add them here as you need them

% define commands here

\begin{document}
Want to extend a homeomorphism of the circle $S^1$ to the whole disk $D^2$?

Let $f\colon S^1\to S^1$ be a homeomorphism. Then the formula 
$$F(x)=||x||f(x/||x||)$$
allows you to define a map $F\colon D^2\to D^2$
which extends $f$, for if $x\in S^1\subset D^2$ then $||x||=1$ and $F(x)=1\cdot f(x/1)=f(x)$. Clearly this map is continuous, save (maybe) the origin, since this formula is undefined there. Nevertheless this is removable.

To check continuity at the origin use: {\it ``A map $f$ is continuous at a point $p$ if and only if  for each sequence $x_n\to p$, $f(x_n)\to f(p)$''}.

So take a sequence $u_n\in D^2$ such that $u_n\to 0$ (i.e. which tends to the origin). Then $F(u_n)=||u_n||f(u_n/||u_n||)$ and since $f(u_n/||u_n||)\neq 0$, hence $||u_n||\to 0$ implies $F(u_n)\to 0$, that is $F$ is also continuous at the origin. 

The same method works for $f^{-1}$.

In the same vein one can extend homeomorphisms $S^n\to S^n$ to $D^{n+1}\to D^{n+1}$.
%%%%%
%%%%%
\end{document}
