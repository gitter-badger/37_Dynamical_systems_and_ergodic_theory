\documentclass[12pt]{article}
\usepackage{pmmeta}
\pmcanonicalname{ErgodicTheorem}
\pmcreated{2013-03-22 12:20:52}
\pmmodified{2013-03-22 12:20:52}
\pmowner{Koro}{127}
\pmmodifier{Koro}{127}
\pmtitle{ergodic theorem}
\pmrecord{11}{31996}
\pmprivacy{1}
\pmauthor{Koro}{127}
\pmtype{Theorem}
\pmcomment{trigger rebuild}
\pmclassification{msc}{37A30}
\pmclassification{msc}{47A35}
\pmsynonym{strong ergodic theorem}{ErgodicTheorem}
\pmsynonym{Birkhoff ergodic theorem}{ErgodicTheorem}
\pmsynonym{Birkhoff-Khinchin ergodic theorem}{ErgodicTheorem}
%\pmkeywords{ergodic}
\pmrelated{ErgodicTransformation}

\usepackage{amssymb}
\usepackage{amsmath}
\usepackage{amsfonts}

% used for TeXing text within eps files
%\usepackage{psfrag}
% need this for including graphics (\includegraphics)
%\usepackage{graphicx}
% for neatly defining theorems and propositions
%\usepackage{amsthm}
% making logically defined graphics
%%%\usepackage{xypic} 

% there are many more packages, add them here as you need them

% define commands here
\newcommand{\mv}[1]{\mathbf{#1}}	% matrix or vector
\newcommand{\mvt}[1]{\mv{#1}^{\mathrm{T}}}
\newcommand{\mvi}[1]{\mv{#1}^{-1}}
\newcommand{\mpderiv}[1]{\frac{\partial}{\partial {#1}}}
\newcommand{\borel}{\mathfrak{B}}
\newcommand{\reals}{\mathbb{R}}
\newcommand{\defined}{:=}
\newcommand{\var}{\mathrm{var}}
\newcommand{\cov}{\mathrm{cov}}
\newcommand{\corr}{\mathrm{corr}}
\newcommand{\set}[1]{\{#1\}}
\begin{document}
Let $(X, \borel, \mu)$ be a probability space, $f \in L^1(\mu)$, and $T\colon X \to X$ a measure preserving transformation.  Birkhoff's \emph{ergodic theorem} (often called the \emph{pointwise} or \emph{strong} ergodic theorem) states that there exists $f^*\in L^1(\mu)$ such that
\begin{equation*}
\lim_{n\to\infty}\frac{1}{n}\sum_{k=0}^{n-1} f(T^k x) = f^*(x)
\end{equation*}
for almost all $x\in X$. Moreover, $f^*$ is $T$-invariant (i.e., $f^*\circ T = f^*$) almost everywhere and $$\int f^*d\mu = \int f d\mu.$$ 
In particular, if $T$ is ergodic then the $T$-invariance of $f^*$ implies that it is constant almost everywhere, and so this constant must be the integral of $f^*$; that is, if $T$ is ergodic, then 
$$\lim_{n\to\infty}\frac{1}{n}\sum_{k=0}^{n-1} f(T^k x) = \int fd\mu$$ 
for almost every $x$. This is often interpreted in the following way: for an ergodic transformation, the time average equals the space average almost surely.
%%%%%
%%%%%
\end{document}
