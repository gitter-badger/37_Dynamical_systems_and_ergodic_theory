\documentclass[12pt]{article}
\usepackage{pmmeta}
\pmcanonicalname{PughsGeneralDensityTheorem}
\pmcreated{2013-03-22 13:40:42}
\pmmodified{2013-03-22 13:40:42}
\pmowner{Koro}{127}
\pmmodifier{Koro}{127}
\pmtitle{Pugh's general density theorem}
\pmrecord{8}{34345}
\pmprivacy{1}
\pmauthor{Koro}{127}
\pmtype{Theorem}
\pmcomment{trigger rebuild}
\pmclassification{msc}{37C20}
\pmclassification{msc}{37C25}
\pmsynonym{general density theorem}{PughsGeneralDensityTheorem}

% this is the default PlanetMath preamble.  as your knowledge
% of TeX increases, you will probably want to edit this, but
% it should be fine as is for beginners.

% almost certainly you want these
\usepackage{amssymb}
\usepackage{amsmath}
\usepackage{amsfonts}
\usepackage{mathrsfs}

% used for TeXing text within eps files
%\usepackage{psfrag}
% need this for including graphics (\includegraphics)
%\usepackage{graphicx}
% for neatly defining theorems and propositions
%\usepackage{amsthm}
% making logically defined graphics
%%%\usepackage{xypic}

% there are many more packages, add them here as you need them

% define commands here
\newcommand{\C}{\mathbb{C}}
\newcommand{\R}{\mathbb{R}}
\newcommand{\N}{\mathbb{N}}
\newcommand{\Z}{\mathbb{Z}}
\newcommand{\Per}{\operatorname{Per}}
\newcommand{\Diff}{\operatorname{Diff}}
\newcommand{\Cdiff}{\mathcal{C}}
\begin{document}
Let $M$ be a compact smooth manifold. There is a residual subset of $\Diff^1(M)$ in which every element $f$ satisfies $\overline{\Per(f)}= \Omega(f)$.
In other words: Generically, the set of periodic points of a $\Cdiff^1$ diffeomorphism is dense in its nonwandering set.

Here, $\Diff^1(M)$ denotes the set of all $\Cdiff^1$ difeomorphisms from $M$ to 
itself, endowed with the (strong) $\Cdiff^1$ topology.

\begin{thebibliography}{9}
\bibitem{Pugh} Pugh, C., \emph{An improved closing lemma and a general density theorem}, Amer. J. Math. \textbf{89} (1967).
\end{thebibliography}
%%%%%
%%%%%
\end{document}
