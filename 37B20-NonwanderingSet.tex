\documentclass[12pt]{article}
\usepackage{pmmeta}
\pmcanonicalname{NonwanderingSet}
\pmcreated{2013-03-22 13:39:31}
\pmmodified{2013-03-22 13:39:31}
\pmowner{Koro}{127}
\pmmodifier{Koro}{127}
\pmtitle{nonwandering set}
\pmrecord{4}{34314}
\pmprivacy{1}
\pmauthor{Koro}{127}
\pmtype{Definition}
\pmcomment{trigger rebuild}
\pmclassification{msc}{37B20}
\pmrelated{OmegaLimitSet3}
\pmrelated{RecurrentPoint}
\pmdefines{wandering point}
\pmdefines{nonwandering point}

\endmetadata

% this is the default PlanetMath preamble.  as your knowledge
% of TeX increases, you will probably want to edit this, but
% it should be fine as is for beginners.

% almost certainly you want these
\usepackage{amssymb}
\usepackage{amsmath}
\usepackage{amsfonts}
\usepackage{mathrsfs}

% used for TeXing text within eps files
%\usepackage{psfrag}
% need this for including graphics (\includegraphics)
%\usepackage{graphicx}
% for neatly defining theorems and propositions
%\usepackage{amsthm}
% making logically defined graphics
%%%\usepackage{xypic}

% there are many more packages, add them here as you need them

% define commands here
\newcommand{\C}{\mathbb{C}}
\newcommand{\R}{\mathbb{R}}
\newcommand{\N}{\mathbb{N}}
\newcommand{\Z}{\mathbb{Z}}
\begin{document}
Let $X$ be a metric space, and $f:X\rightarrow X$ a continuous surjection.
An element $x$ of $X$ is a \emph{wandering point} if there is a neighborhood $U$ of $x$ and an integer $N$ such that, for all $n\geq N$, $f^n(U)\cap U=\emptyset$. If $x$ is not wandering, we call it a \emph{nonwandering point}. Equivalently, $x$ is a nonwandering point if for every neighborhood $U$
of $x$ there is $n\geq 1$ such that $f^n(U)\cap U$ is nonempty. The set of all nonwandering points is called the \emph{nonwandering set} of $f$, and is denoted by $\Omega(f)$.

If $X$ is compact, then $\Omega(f)$ is compact, nonempty, and forward invariant; if, additionally, $f$ is an homeomorphism, then $\Omega(f)$ is invariant.
%%%%%
%%%%%
\end{document}
