\documentclass[12pt]{article}
\usepackage{pmmeta}
\pmcanonicalname{SharkovskiisTheorem}
\pmcreated{2013-03-22 13:16:11}
\pmmodified{2013-03-22 13:16:11}
\pmowner{Koro}{127}
\pmmodifier{Koro}{127}
\pmtitle{Sharkovskii's theorem}
\pmrecord{7}{33751}
\pmprivacy{1}
\pmauthor{Koro}{127}
\pmtype{Definition}
\pmcomment{trigger rebuild}
\pmclassification{msc}{37E05}
\pmsynonym{Sharkovsky's theorem}{SharkovskiisTheorem}
\pmdefines{Sharkovskii's ordering}
\pmdefines{Sharkovsky's theorem}

% this is the default PlanetMath preamble.  as your knowledge
% of TeX increases, you will probably want to edit this, but
% it should be fine as is for beginners.

% almost certainly you want these
\usepackage{amssymb}
\usepackage{amsmath}
\usepackage{amsfonts}

% used for TeXing text within eps files
%\usepackage{psfrag}
% need this for including graphics (\includegraphics)
%\usepackage{graphicx}
% for neatly defining theorems and propositions
%\usepackage{amsthm}
% making logically defined graphics
%%%\usepackage{xypic}

% there are many more packages, add them here as you need them

% define commands here
\begin{document}
Every natural number can be written as $2^rp$, where $p$ is odd, and $r$ is the maximum exponent such that $2^r$ \PMlinkname{divides}{Divisibility} the given number. We define the \emph{Sharkovskii ordering} of the natural numbers in this way: given two odd numbers $p$ and $q$, and two nonnegative integers $r$ and $s$, 
then $2^rp\succ 2^sq$ if
\begin{enumerate}
\item $r<s$ and $p>1$;
\item $r=s$ and $p<q$;
\item $r>s$ and $p=q=1$.
\end{enumerate}
This defines a linear ordering of $\mathbb{N}$, in which we first have $3,5,7,\dots$, followed by $2\cdot 3$, $2\cdot 5,\dots$, 
followed by $2^2\cdot 3$, $2^2\cdot 5,\dots$,
and so on, and finally $2^{n+1},2^n,\dots,2,1$. So it looks like this:
\[3\succ 5 \succ\cdots\succ 3\cdot 2\succ 5\cdot2\succ\cdots\succ 
3\cdot2^n\succ 5\cdot 2^n \succ\cdots\succ 2^2\succ 2\succ 1.\]

\textbf{Sharkovskii's theorem.} Let $I\subset \mathbb{R}$ be an interval, and let $f:I\rightarrow \mathbb{R}$ be a continuous function. If $f$ has a periodic point of least period $n$, then $f$ has a periodic point of least period $k$, for each $k$ such that $n\succ k$.
%%%%%
%%%%%
\end{document}
