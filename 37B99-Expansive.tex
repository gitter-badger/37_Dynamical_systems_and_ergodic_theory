\documentclass[12pt]{article}
\usepackage{pmmeta}
\pmcanonicalname{Expansive}
\pmcreated{2013-03-22 13:47:48}
\pmmodified{2013-03-22 13:47:48}
\pmowner{Koro}{127}
\pmmodifier{Koro}{127}
\pmtitle{expansive}
\pmrecord{14}{34513}
\pmprivacy{1}
\pmauthor{Koro}{127}
\pmtype{Definition}
\pmcomment{trigger rebuild}
\pmclassification{msc}{37B99}
\pmdefines{expansivity}
\pmdefines{positively expansive}
\pmdefines{forward expansive}

% this is the default PlanetMath preamble.  as your knowledge
% of TeX increases, you will probably want to edit this, but
% it should be fine as is for beginners.

% almost certainly you want these
\usepackage{amssymb}
\usepackage{amsmath}
\usepackage{amsfonts}
\usepackage{mathrsfs}

% used for TeXing text within eps files
%\usepackage{psfrag}
% need this for including graphics (\includegraphics)
%\usepackage{graphicx}
% for neatly defining theorems and propositions
%\usepackage{amsthm}
% making logically defined graphics
%%%\usepackage{xypic}

% there are many more packages, add them here as you need them

% define commands here
\newcommand{\C}{\mathbb{C}}
\newcommand{\R}{\mathbb{R}}
\newcommand{\N}{\mathbb{N}}
\newcommand{\Z}{\mathbb{Z}}
\newcommand{\Per}{\operatorname{Per}}
\begin{document}
If $(X,d)$ is a metric space, a homeomorphism $f\colon X\to X$ is said to be \textbf{expansive} if there is a constant $\varepsilon_0>0$, called the \textbf{expansivity constant}, such that for any two points of $X$, their $n$-th iterates are at least $\varepsilon_0$ apart for some integer $n$; i.e. if for any pair of points $x\neq y$ in $X$ there is $n\in \Z$ such that $d(f^n(x),f^n(y))\geq \varepsilon_0$.  

The space $X$ is often assumed to be compact, since under that assumption expansivity is a topological property; i.e. any map which is topologically conjugate to $f$ is expansive if $f$ is expansive (possibly with a different expansivity constant).

If $f\colon X\to X$ is a continuous map, we say that $X$ is \textbf{positively expansive} (or forward expansive) if there is $\varepsilon_0$ such that, for any $x\neq y$ in $X$, there is $n\in \N$ such that $d(f^n(x),f^n(y))\geq \varepsilon_0$. 

\textbf{Remarks.} The latter condition is much stronger than expansivity. In fact, one can prove that if $X$ is compact and $f$ is a positively expansive homeomorphism, then $X$ is finite (\PMlinkname{proof}{OnlyCompactMetricSpacesThatAdmitAPostivelyExpansiveHomeomorphismAreDiscreteSpaces}).
%%%%%
%%%%%
\end{document}
