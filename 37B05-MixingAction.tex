\documentclass[12pt]{article}
\usepackage{pmmeta}
\pmcanonicalname{MixingAction}
\pmcreated{2013-03-22 19:19:32}
\pmmodified{2013-03-22 19:19:32}
\pmowner{Ziosilvio}{18733}
\pmmodifier{Ziosilvio}{18733}
\pmtitle{mixing action}
\pmrecord{5}{42265}
\pmprivacy{1}
\pmauthor{Ziosilvio}{18733}
\pmtype{Definition}
\pmcomment{trigger rebuild}
\pmclassification{msc}{37B05}

% this is the default PlanetMath preamble.  as your knowledge
% of TeX increases, you will probably want to edit this, but
% it should be fine as is for beginners.

% almost certainly you want these
\usepackage{amssymb}
\usepackage{amsmath}
\usepackage{amsfonts}

% used for TeXing text within eps files
%\usepackage{psfrag}
% need this for including graphics (\includegraphics)
%\usepackage{graphicx}
% for neatly defining theorems and propositions
%\usepackage{amsthm}
% making logically defined graphics
%%%\usepackage{xypic}

% there are many more packages, add them here as you need them

% define commands here

\begin{document}
\newcommand{\restrict}[2]{\left.{#1}\right|_{#2}}

Let $X$ be a topological space and let $G$ be a semigroup.
An action $\phi=\{\phi_g\}_{g \in G}$ of $G$ on $X$ is (topologically) mixing
if, given any two \emph{open} subsets $U$, $V$ of $X$,
the intersection $U \cap \phi_g(V)$ is nonempty
for all $g \in G$ except at most finitely many.

\medskip

\textbf{Example 1.}
Let $F : X \to X$ be a continuous function.
Then $F$ is topologically mixing if and only if
the action of the monoid $\mathbb{N}$ on $X$
defined by $\phi_n(x) = F^n(x)$
is mixing according to the definition given above.

\medskip

\textbf{Example 2.}
Suppose $X$ is a discrete nonempty set and $G$ is a group;
endow $X^G$ with the product topology.
The action of $G$ on $X^G$ defined by
\begin{displaymath}
\phi_g(c)(z) = c(g \cdot z) \;\; \forall c : G \to X
\end{displaymath}
is mixing.

To prove this fact,
we may suppose without loss of generality
that $U$ and $V$ are two \emph{cylindric sets}
of the form:
\begin{eqnarray*}
U & = & \{ c \in X^G \mid \restrict{c}{E} = \restrict{u}{E} \} \\
V & = & \{ c \in X^G \mid \restrict{c}{F} = \restrict{v}{F} \}
\end{eqnarray*}
for suitable finite subsets $E,F \subseteq G$ and functions $u,v : G \to X$.
Then the only chance for $U \cap \phi_g(V)$ to be empty,
is that $e = gf$ for some $e \in E$, $f \in F$
such that $u(e) \neq v(f)$:
but then, $g \in EF^{-1}$, which is finite.
%%%%%
%%%%%
\end{document}
